\documentclass[]{book}
\usepackage{lmodern}
\usepackage{amssymb,amsmath}
\usepackage{ifxetex,ifluatex}
\usepackage{fixltx2e} % provides \textsubscript
\ifnum 0\ifxetex 1\fi\ifluatex 1\fi=0 % if pdftex
  \usepackage[T1]{fontenc}
  \usepackage[utf8]{inputenc}
\else % if luatex or xelatex
  \ifxetex
    \usepackage{mathspec}
  \else
    \usepackage{fontspec}
  \fi
  \defaultfontfeatures{Ligatures=TeX,Scale=MatchLowercase}
\fi
% use upquote if available, for straight quotes in verbatim environments
\IfFileExists{upquote.sty}{\usepackage{upquote}}{}
% use microtype if available
\IfFileExists{microtype.sty}{%
\usepackage{microtype}
\UseMicrotypeSet[protrusion]{basicmath} % disable protrusion for tt fonts
}{}
\usepackage{hyperref}
\hypersetup{unicode=true,
            pdftitle={Don't Sweat It: A Guide to Dealing with Excessive Sweating},
            pdfauthor={cogsciguy},
            pdfborder={0 0 0},
            breaklinks=true}
\urlstyle{same}  % don't use monospace font for urls
\usepackage{natbib}
\bibliographystyle{apalike}
\usepackage{longtable,booktabs}
\usepackage{graphicx,grffile}
\makeatletter
\def\maxwidth{\ifdim\Gin@nat@width>\linewidth\linewidth\else\Gin@nat@width\fi}
\def\maxheight{\ifdim\Gin@nat@height>\textheight\textheight\else\Gin@nat@height\fi}
\makeatother
% Scale images if necessary, so that they will not overflow the page
% margins by default, and it is still possible to overwrite the defaults
% using explicit options in \includegraphics[width, height, ...]{}
\setkeys{Gin}{width=\maxwidth,height=\maxheight,keepaspectratio}
\IfFileExists{parskip.sty}{%
\usepackage{parskip}
}{% else
\setlength{\parindent}{0pt}
\setlength{\parskip}{6pt plus 2pt minus 1pt}
}
\setlength{\emergencystretch}{3em}  % prevent overfull lines
\providecommand{\tightlist}{%
  \setlength{\itemsep}{0pt}\setlength{\parskip}{0pt}}
\setcounter{secnumdepth}{5}
% Redefines (sub)paragraphs to behave more like sections
\ifx\paragraph\undefined\else
\let\oldparagraph\paragraph
\renewcommand{\paragraph}[1]{\oldparagraph{#1}\mbox{}}
\fi
\ifx\subparagraph\undefined\else
\let\oldsubparagraph\subparagraph
\renewcommand{\subparagraph}[1]{\oldsubparagraph{#1}\mbox{}}
\fi

%%% Use protect on footnotes to avoid problems with footnotes in titles
\let\rmarkdownfootnote\footnote%
\def\footnote{\protect\rmarkdownfootnote}

%%% Change title format to be more compact
\usepackage{titling}

% Create subtitle command for use in maketitle
\providecommand{\subtitle}[1]{
  \posttitle{
    \begin{center}\large#1\end{center}
    }
}

\setlength{\droptitle}{-2em}

  \title{Don't Sweat It: A Guide to Dealing with Excessive Sweating}
    \pretitle{\vspace{\droptitle}\centering\huge}
  \posttitle{\par}
    \author{cogsciguy}
    \preauthor{\centering\large\emph}
  \postauthor{\par}
      \predate{\centering\large\emph}
  \postdate{\par}
    \date{2019-09-11}

\usepackage{booktabs}
\usepackage{amsthm}
\makeatletter
\def\thm@space@setup{%
  \thm@preskip=8pt plus 2pt minus 4pt
  \thm@postskip=\thm@preskip
}
\makeatother

\begin{document}
\maketitle

{
\setcounter{tocdepth}{1}
\tableofcontents
}
\hypertarget{intro}{%
\chapter{Introduction: Why we sweat}\label{intro}}

\hypertarget{a-human-history-of-sweat}{%
\section{A human history of sweat}\label{a-human-history-of-sweat}}

Sweat sets us apart from most animals: humans have the most pervasive sweat (eccrine) system.
Why do we need such an cooling system?
One hypothesis is that our ancestors were persistence hunters, meaning that they would chase and track prey like gazelles for long distances on the savannah.
Although we cannot outrun gazelles and other prey for short distances, we have more stamina, and they will become exhausted after running several miles, scared for their lives.
After tracking their prey and eventually catching up, our ancestors may have only had to deliver a coups de grace to the frightened beasts.

Could diversity in our ancestors' climates or diets have further impacted our present sweating behavior?

\hypertarget{how-much-sweat-is-too-much}{%
\section{How much sweat is too much?}\label{how-much-sweat-is-too-much}}

\begin{verbatim}
- normal sweating vs. hyperhidrosis
- social impact of too much sweating
- axillary hyperhidrosis
- palmar-plantar hyperhidrosis
\end{verbatim}

\hypertarget{anecdotes}{%
\section{Anecdotes}\label{anecdotes}}

After finding relief for my sweaty hands, I took up social dancing and became more aware of how many others suffer from sweaty hands in the US: I estimate as many as 1 in 15 of my dance partners would have at least mildly sweaty palms.
And just think--those are the ones who still chose to go dancing!
However, I was more surprised after attending a swing dance event in Stockholm, Sweden, where I found that nearly half of the partners I danced with had hands that rivaled mine in sweatiness.
Granted, this was a single event and I danced with no more than a dozen partners, but this suspicious coincidence of so many dancers suffering HH in Stockholm, along with my family's northern European heritage led me to the hypothesis that HH may be more prevalent in some populations -- a genetic basis for HH.

Coupled with this observation is the fact that a few women of Mediterranean descent I've known sweat too little (and instead become red-faced).

\hypertarget{outline-of-the-book}{%
\section{Outline of the book}\label{outline-of-the-book}}

The outline of this book is as follows.
In \texttt{\{\#science\}} the basic science of sweating is explained, from how sweat glands work to the nervous system that controls them.
In Chapter \ref{intro}. If you do not manually label them, there will be automatic labels anyway, e.g., Chapter \ref{methods}.

\hypertarget{science}{%
\chapter{The Science of Sweating}\label{science}}

\hypertarget{physiology-the-sweat-gland}{%
\section{Physiology: The sweat gland}\label{physiology-the-sweat-gland}}

\hypertarget{the-sympathetic-nervous-system}{%
\section{The sympathetic nervous system}\label{the-sympathetic-nervous-system}}

\begin{verbatim}
    - acetylcholine
    - cortisol 
    - thyroid
\end{verbatim}

\hypertarget{psychological-influences}{%
\section{Psychological Influences}\label{psychological-influences}}

To be clear, unless you are a dualist, psychology has a physical reality -- something cannot actually be `just in your head' in the sense that it is not real.
But what we mean to explain in this section is that HH in some can be exacerbated by certain social situations, and may be comorbid with anxiety disorders.
However, I and many other HH sufferers sweat abnormally even in the safety and comfort of our home, when stress and anxiety levels are at their lowest.

\hypertarget{diet}{%
\chapter{Diet}\label{diet}}

\begin{verbatim}
- salt
- hydration
- caffeine - often increases sweating
- Sudafed - can increase sweating
- alcohol (many find it reduces sweating: psychogenic?)
- nicotine?
\end{verbatim}

\hypertarget{oral}{%
\chapter{Oral Medications}\label{oral}}

\hypertarget{anticholinergics}{%
\section{Anticholinergics}\label{anticholinergics}}

\hypertarget{glycopyrrolate}{%
\subsection{Glycopyrrolate}\label{glycopyrrolate}}

1-2 mg taken on an empty stomach (wait 2-3 hours to eat), but common side effects include dry mouth / dry eyes, may mix poorly with alcohol (blacking out reported)

\hypertarget{oxybutynin-5-10mg---cons-side-effects}{%
\subsection{Oxybutynin (5-10mg) - cons: side effects}\label{oxybutynin-5-10mg---cons-side-effects}}

\hypertarget{antimuscarinics}{%
\section{Antimuscarinics}\label{antimuscarinics}}

\hypertarget{propantheline-bromide}{%
\subsection{Propantheline bromide}\label{propantheline-bromide}}

Also known as pro-banthine, this antimuscarinic agent is used for the treatment of excessive sweating, cramps or spasms of the stomach, intestines or bladder, and involuntary urination

\hypertarget{antihistamines}{%
\section{Antihistamines}\label{antihistamines}}

\hypertarget{benadryl}{%
\subsection{Benadryl}\label{benadryl}}

\hypertarget{psychiatric-medications}{%
\section{Psychiatric medications}\label{psychiatric-medications}}

\hypertarget{topical}{%
\chapter{Topical Antiperspirants}\label{topical}}

\begin{verbatim}
- standard aluminum hexachloride (driclor, certain dry)
- methenamine (dehydral, antihydral, rhino dry)
    - con: increased skin wrinkling when wet ("pruniness") 
\end{verbatim}

\hypertarget{ionto}{%
\chapter{Iontophoresis}\label{ionto}}

\hypertarget{commercial-units}{%
\section{Commercial units}\label{commercial-units}}

\hypertarget{drionic}{%
\subsection{Drionic}\label{drionic}}

\hypertarget{fischer-md-1a}{%
\subsection{Fischer MD-1a}\label{fischer-md-1a}}

\hypertarget{idrostar}{%
\subsection{Idrostar}\label{idrostar}}

\hypertarget{homemade-units}{%
\section{Homemade units}\label{homemade-units}}

\hypertarget{power-supply}{%
\subsection{Power supply}\label{power-supply}}

lantern batteries, DC adapter (danger!), galvanic pulsed stimulator

\hypertarget{water-reservoir}{%
\subsection{Water reservoir}\label{water-reservoir}}

Plastic bins/trays, aluminum pie pans, etc.

\hypertarget{wiring-tips}{%
\subsection{Wiring tips}\label{wiring-tips}}

buy alligator clips
use aluminum foil/strips as electrodes: they will corrode, so replace regularly

\hypertarget{treatment-schedule}{%
\subsection{Treatment schedule}\label{treatment-schedule}}

Initial - 7 to 30 (!) daily treatments, 20 minutes per appendange; often 6-10 initial treatments
Maintenance - a treatment once every 3 to 14 days (varies per individual: if you required more initial treatments, you will require more frequent maintenance treatments)

\hypertarget{water-type-and-additives}{%
\section{Water type and additives}\label{water-type-and-additives}}

\hypertarget{know-your-tap-water}{%
\subsection{Know your tap water}\label{know-your-tap-water}}

Trace minerals seem to vastly affect how well iontophoresis works.
Do you know if your tap water is hard (containing minerals such as iron and calcium) or soft (low mineral content)?
Several users have found that hard water works better than soft or distilled water (which contains no minerals).
Note that not all hard water is the same: some may have high iron content, while others are relatively higher in magnesium, calcium, or other minerals.
We do not yet know what trace mineral(s) increase the efficacy of iontophoresis--or if they are indeed critical.
If you're unsure how hard your tap water is, an alternative to tap water that some users have found effective is to use rain water (which also contains an array of trace minerals that is likely more consistent worldwide than tap water).
If rain water is infeasible to collect and your tap water doesn't seem effective, don't give up hope: various additives have been found to increase iontophoresis efficacy.

\hypertarget{salt}{%
\subsection{Salt}\label{salt}}

Iodized table salt vs.~sea salt with trace minerals (magnesium, potassium, and more)

\hypertarget{baking-soda}{%
\subsection{Baking soda}\label{baking-soda}}

Sodium bicarbonate

\hypertarget{antiperspirants}{%
\subsection{Antiperspirants}\label{antiperspirants}}

Some users find that the effectiveness of tap water iontophoresis is vastly improved (i.e., less treatment needed to achieve the same reduction in sweat) if they first apply an aluminum-based antiperspirant such as Driclor or Certain Dri to their hands/feet.

\hypertarget{final-words}{%
\chapter{Final Words}\label{final-words}}

\hypertarget{to-others-with-hyperhidrosis}{%
\section{To others with hyperhidrosis}\label{to-others-with-hyperhidrosis}}

In comparison to many other medical conditions, hyperhidrosis is quite under-researched and not well understood, and I hope we can work together to help change that.
Towards that end, I invite you to participate in an ongoing anonymous informal survey about your experience with hyperhidrosis:
\url{https://forms.gle/KerqNZ2k54Lk3EzB6}
The aggregated results of this survey will be used to continuously update the contents of this book.

\hypertarget{to-our-friends-and-family}{%
\section{To our friends and family}\label{to-our-friends-and-family}}

We thank you for taking our concerns about excessive sweating seriously, and for being patient with us as we try to figure out ways to mitigate its effects.

\hypertarget{to-the-medical-community}{%
\section{To the medical community}\label{to-the-medical-community}}

\hypertarget{additional-resources}{%
\section{Additional Resources:}\label{additional-resources}}

\begin{verbatim}
- hyperhidrosis forums:
    https://socialphobiaworld.com/forums/hyperhidrosis-forum.22/
    https://www.reddit.com/r/Hyperhidrosis/
\end{verbatim}

\hypertarget{acknowledgements}{%
\section{Acknowledgements}\label{acknowledgements}}

\bibliography{book.bib,packages.bib}


\end{document}
